\documentclass[10pt]{article}
\usepackage{amsmath}
\usepackage{bm}
\usepackage{bbm}
\usepackage{mathrsfs}
\usepackage{graphicx}
\usepackage{wrapfig}
\usepackage{subcaption}
\usepackage{epsfig}
\usepackage{amsfonts}
\usepackage{amssymb}
\usepackage{amsmath}
\usepackage{wrapfig}
\usepackage{graphicx}
\usepackage{psfrag}
\newcommand{\sun}{\ensuremath{\odot}} % sun symbol is \sun
\let\vaccent=\v % rename builtin command \v{} to \vaccent{}
\renewcommand{\v}[1]{\ensuremath{\mathbf{#1}}} % for vectors
\newcommand{\gv}[1]{\ensuremath{\mbox{\boldmath$ #1 $}}} 
\newcommand{\grad}[1]{\gv{\nabla} #1}
\renewcommand{\baselinestretch}{1.2}
\jot 5mm
\graphicspath{{./figures/}}
%text dimensions
\textwidth 6.5 in
\oddsidemargin .2 in
\topmargin -0.2 in
\textheight 8.5 in
\headheight 0.2in
\overfullrule = 0pt
\pagestyle{plain}
\def\newpar{\par\vskip 0.5cm}
\begin{document}
%
%----------------------------------------------------------------------
%        Define symbols
%----------------------------------------------------------------------
%
\def\iso{\mathbbm{1}}
\def\half{{\textstyle{1 \over 2}}}
\def\third{{\textstyle{1 \over 3}}}
\def\fourth{{\textstyle{{1 \over 4}}}}
\def\twothird{{\textstyle {{2 \over 3}}}}
\def\ndim{{n_{\rm dim}}}
\def\nint{n_{\rm int}}
\def\lint{l_{\rm int}}
\def\nel{n_{\rm el}}
\def\nf{n_{\rm f}}
\def\DIV {\hbox{\af div}}
\def\GRAD{\hbox{\af Grad}}
\def\sym{\mathop{\rm sym}\nolimits}
\def\tr{\mathop{\rm tr}\nolimits}
\def\dev{\mathop{\rm dev}\nolimits}
\def\Dev{\mathop{\rm Dev}\nolimits}
\def\DEV{\mathop {\rm DEV}\nolimits}
\def\bfb {{\bi b}}
\def\Bnabla{\nabla}
\def\bG{{\bi G}}
\def\jmpdelu{{\lbrack\!\lbrack \Delta u\rbrack\!\rbrack}}
\def\jmpudot{{\lbrack\!\lbrack\dot u\rbrack\!\rbrack}}
\def\jmpu{{\lbrack\!\lbrack u\rbrack\!\rbrack}}
\def\jmphi{{\lbrack\!\lbrack\varphi\rbrack\!\rbrack}}
\def\ljmp{{\lbrack\!\lbrack}}
\def\rjmp{{\rbrack\!\rbrack}}
\def\sign{{\rm sign}}
\def\nn{{n+1}}
\def\na{{n+\vartheta}}
\def\nna{{n+(1-\vartheta)}}
\def\nt{{n+{1\over 2}}}
\def\nb{{n+\beta}}
\def\nbb{{n+(1-\beta)}}
%---------------------------------------------------------
%               Bold Face Math Characters:
%               All In Format: \B***** .
%---------------------------------------------------------
\def\bOne{\mbox{\boldmath$1$}}
\def\BGamma{\mbox{\boldmath$\Gamma$}}
\def\BDelta{\mbox{\boldmath$\Delta$}}
\def\BTheta{\mbox{\boldmath$\Theta$}}
\def\BLambda{\mbox{\boldmath$\Lambda$}}
\def\BXi{\mbox{\boldmath$\Xi$}}
\def\BPi{\mbox{\boldmath$\Pi$}}
\def\BSigma{\mbox{\boldmath$\Sigma$}}
\def\BUpsilon{\mbox{\boldmath$\Upsilon$}}
\def\BPhi{\mbox{\boldmath$\Phi$}}
\def\BPsi{\mbox{\boldmath$\Psi$}}
\def\BOmega{\mbox{\boldmath$\Omega$}}
\def\Balpha{\mbox{\boldmath$\alpha$}}
\def\Bbeta{\mbox{\boldmath$\beta$}}
\def\Bgamma{\mbox{\boldmath$\gamma$}}
\def\Bdelta{\mbox{\boldmath$\delta$}}
\def\Bepsilon{\mbox{\boldmath$\epsilon$}}
\def\Bzeta{\mbox{\boldmath$\zeta$}}
\def\Beta{\mbox{\boldmath$\eta$}}
\def\Btheta{\mbox{\boldmath$\theta$}}
\def\Biota{\mbox{\boldmath$\iota$}}
\def\Bkappa{\mbox{\boldmath$\kappa$}}
\def\Blambda{\mbox{\boldmath$\lambda$}}
\def\Bmu{\mbox{\boldmath$\mu$}}
\def\Bnu{\mbox{\boldmath$\nu$}}
\def\Bxi{\mbox{\boldmath$\xi$}}
\def\Bpi{\mbox{\boldmath$\pi$}}
\def\Brho{\mbox{\boldmath$\rho$}}
\def\Bsigma{\mbox{\boldmath$\sigma$}}
\def\Btau{\mbox{\boldmath$\tau$}}
\def\Bupsilon{\mbox{\boldmath$\upsilon$}}
\def\Bphi{\mbox{\boldmath$\phi$}}
\def\Bchi{\mbox{\boldmath$\chi$}}
\def\Bpsi{\mbox{\boldmath$\psi$}}
\def\Bomega{\mbox{\boldmath$\omega$}}
\def\Bvarepsilon{\mbox{\boldmath$\varepsilon$}}
\def\Bvartheta{\mbox{\boldmath$\vartheta$}}
\def\Bvarpi{\mbox{\boldmath$\varpi$}}
\def\Bvarrho{\mbox{\boldmath$\varrho$}}
\def\Bvarsigma{\mbox{\boldmath$\varsigma$}}
\def\Bvarphi{\mbox{\boldmath$\varphi$}}
\def\bone{\mathbf{1}}
\def\bzero{\mathbf{0}}
%---------------------------------------------------------
%               Bold Face Math Italic:
%               All In Format: \b* .
%---------------------------------------------------------
\def\bA{\mbox{\boldmath$ A$}}
\def\bB{\mbox{\boldmath$ B$}}
\def\bC{\mbox{\boldmath$ C$}}
\def\bD{\mbox{\boldmath$ D$}}
\def\bE{\mbox{\boldmath$ E$}}
\def\bF{\mbox{\boldmath$ F$}}
\def\bG{\mbox{\boldmath$ G$}}
\def\bH{\mbox{\boldmath$ H$}}
\def\bI{\mbox{\boldmath$ I$}}
\def\bJ{\mbox{\boldmath$ J$}}
\def\bK{\mbox{\boldmath$ K$}}
\def\bL{\mbox{\boldmath$ L$}}
\def\bM{\mbox{\boldmath$ M$}}
\def\bN{\mbox{\boldmath$ N$}}
\def\bO{\mbox{\boldmath$ O$}}
\def\bP{\mbox{\boldmath$ P$}}
\def\bQ{\mbox{\boldmath$ Q$}}
\def\bR{\mbox{\boldmath$ R$}}
\def\bS{\mbox{\boldmath$ S$}}
\def\bT{\mbox{\boldmath$ T$}}
\def\bU{\mbox{\boldmath$ U$}}
\def\bV{\mbox{\boldmath$ V$}}
\def\bW{\mbox{\boldmath$ W$}}
\def\bX{\mbox{\boldmath$ X$}}
\def\bY{\mbox{\boldmath$ Y$}}
\def\bZ{\mbox{\boldmath$ Z$}}
\def\ba{\mbox{\boldmath$ a$}}
\def\bb{\mbox{\boldmath$ b$}}
\def\bc{\mbox{\boldmath$ c$}}
\def\bd{\mbox{\boldmath$ d$}}
\def\be{\mbox{\boldmath$ e$}}
\def\bff{\mbox{\boldmath$ f$}}
\def\bg{\mbox{\boldmath$ g$}}
\def\bh{\mbox{\boldmath$ h$}}
\def\bi{\mbox{\boldmath$ i$}}
\def\bj{\mbox{\boldmath$ j$}}
\def\bk{\mbox{\boldmath$ k$}}
\def\bl{\mbox{\boldmath$ l$}}
\def\bm{\mbox{\boldmath$ m$}}
\def\bn{\mbox{\boldmath$ n$}}
\def\bo{\mbox{\boldmath$ o$}}
\def\bp{\mbox{\boldmath$ p$}}
\def\bq{\mbox{\boldmath$ q$}}
\def\br{\mbox{\boldmath$ r$}}
\def\bs{\mbox{\boldmath$ s$}}
\def\bt{\mbox{\boldmath$ t$}}
\def\bu{\mbox{\boldmath$ u$}}
\def\bv{\mbox{\boldmath$ v$}}
\def\bw{\mbox{\boldmath$ w$}}
\def\bx{\mbox{\boldmath$ x$}}
\def\by{\mbox{\boldmath$ y$}}
\def\bz{\mbox{\boldmath$ z$}}
%*********************************
%Start main paper
%*********************************
\centerline{\Large{\bf PRISMS-PF Application Formulation:}}
\smallskip
\centerline{\Large{\bf additive\_manufacturing}}
\bigskip

This example application implements a simple set of governing equations for isotropic solidification and grain growth, with an accompanying temperature evolution equation. The model is a simplified version of the one in the following publication:\\
 Phase-field modeling of grain evolutions in additive manufacturing from nucleation, growth, to coarsening, Yang, M., Wang, L. \& Yan, W., \emph{npj Comput Mater} 7, 56 (2021).
\\
\\
Consider a free energy expression of the form:
\begin{equation}
F=\int_{\Omega}\left(f_{\mathrm{phase}}+f_{\mathrm{grain}}+f_{\mathrm{gradient}}\right)dV\label{eq:Free_energy_equation}
\end{equation}


where $f_{\mathrm{phase}}$ and $f_{\mathrm{grain}}$ represent contributions to the free energy density due to interfaces between the liquid and the solid phases, and contributions from the grain boundary interfaces, respectively. $f_{\mathrm{gradient}}$ captures the gradient contributions from the inter-grain and inter-state interfaces. 

\begin{equation}
f_{\mathrm{phase}}=m_{p}\left\{ \left(1-\xi\right)^{2}\phi+\xi^{2}\left(1-\phi\right)\right\} \label{eq:f_phase}
\end{equation}

\begin{equation}
f_{\mathrm{grain}}=m_{g} \left\{ \sum_{i=1}^{n}\left(\frac{\eta_{i}^{4}}{4}-\frac{\eta_{i}^{2}}{2}\right)+\gamma\sum_{i=1}^{n}\sum_{j\neq i}\eta_{i}^{2}\eta_{j}^{2}+\frac{1}{4}\left(1-\xi\right)^{2}\sum_{i=1}^{n}\eta_{i}^{2}\right\} \label{eq:f_grain}
\end{equation}

\begin{equation}
f_{\mathrm{gradient}}=\frac{\kappa_{p}}{2}\left(\nabla\xi\right)^{2}+\frac{\kappa_{g}}{2}\left(\nabla\eta_{i}\right)^{2}\label{eq:f_gradient}
\end{equation}


where $\eta_i$ is one of $N$ structural order parameters that describe the solid grains, $\xi$ is am order parameter that describes solid/liquid states; $\xi=0$ in the liquid state and $\xi=1$ in the solid state.  $\gamma$ is the grain interaction coefficient, and $\kappa$ is the gradient energy coefficient.
	
\section{Variational treatment}
The driving force for grain evolution is determined by the variational derivative of the total energy with respect to each order parameter:

\begin{equation}
\mu_g = \frac{\delta F}{\delta \eta_i} = m_g \left( -\eta_i + \eta_i^3 + 2 \gamma \eta_i \sum_{j \ne i}^N \eta_j^2 +2\eta_i\left( 1-\xi \right)^2  \right) - \kappa_g \nabla^2 \eta_i
\end{equation}

\begin{equation}
\mu_p = \frac{\delta F}{\delta \xi} = m_p \left(  -2 \left( 1- \xi \right) \phi +2 \xi \left( 1- \phi \right)  \right) + m_g \left( -2 \left( 1-\xi \right) \sum_{i=1}^{n} \eta_i^2 \right) - \kappa_p \nabla^2 \xi
\end{equation}

\section{Kinetics}
The order parameter for each grain is unconserved, and thus their evolution can be described by Allen-Cahn equations:

\begin{equation}
\frac{\partial \eta_i}{\partial t} = -L_g \mu_g = L_g \left( m_g \left( -\eta_i + \eta_i^3 + 2 \gamma \eta_i \sum_{j \ne i}^N \eta_j^2 +2\eta_i\left( 1-\xi \right)^2  \right) - \kappa_g \nabla^2 \eta_i \right)
\end{equation}
where $L_g$ is the constant grain boundary mobility. 

Similarly, the liquis/solid order parameter, $\xi$, is unconserved:
\begin{equation}
\frac{\partial \xi}{\partial t} = -L_p \mu_p = -L_g \left( m_p \left(  -2 \left( 1- \xi \right) \phi +2 \xi \left( 1- \phi \right)  \right) + m_g \left( -2 \left( 1-\xi \right) \sum_{i=1}^{n} \eta_i^2 \right) - \kappa_p \nabla^2 \xi \right)
\end{equation}

Additionally in this application, a transient temperature evolution equation (Heat Equation) is solved:

\begin{equation}
\frac{\partial T}{\partial t} = -\grad \cdot (-D~\grad T) 
\end{equation}

Were $D$ is the thermal diffusivity


\section{Time discretization}
Considering forward Euler explicit time stepping, we have the time discretized kinetics equation:
\begin{align}
 \eta_i^{n+1} &= \eta_i^{n} - \Delta t L_g~\left( m_g \left( -\eta_i + \eta_i^3 + 2 \gamma \eta_i \sum_{j \ne i}^N \eta_j^2 +2\eta_i\left( 1-\xi \right)^2  \right) - \kappa_g \nabla^2 \eta_i \right)
\end{align}


 \begin{align}
 \xi^{n+1} &= \xi^{n} - \Delta t L_p~\left( m_p \left(  -2 \left( 1- \xi \right) \phi +2 \xi \left( 1- \phi \right)  \right) + m_g \left( -2 \left( 1-\xi \right) \sum_{i=1}^{n} \eta_i^2 \right) - \kappa_p \nabla^2 \xi \right)
\end{align}

 \begin{align}
 T^{n+1} &= T^{n} + (\Delta t D)~\Delta T^n 
\end{align}

\section{Weak formulation}
In the weak formulation, considering an arbitrary variation $w$, the above equation can be expressed as a residual equation:

\begin{align}
\int_{\Omega}   w \eta_i^{n+1} ~dV&= \int_{\Omega}   w \eta_i^{n} - w \Delta t L_g~\left( m_g \left( -\eta_i + \eta_i^3 + 2 \gamma \eta_i \sum_{j \ne i}^N \eta_j^2 +2\eta_i\left( 1-\xi \right)^2  \right) - \kappa_g \nabla^2 \eta_i \right)~dV \\
&= \int_{\Omega}   w ( \underbrace{ \eta^{n} - \Delta t L_g~m_g \left( -\eta_i + \eta_i^3 + 2 \gamma \eta_i \sum_{j \ne i}^N \eta_j^2 +2\eta_i\left( 1-\xi \right)^2  \right) }_{r_{\eta_i}} + \grad w \underbrace{ (-\Delta t L_g \kappa_g)~ \cdot (\grad \eta_i^{n})}_{r_{\eta_i x}} ~dV \quad [\kappa_g \grad \eta_i \cdot n = 0 ~ \text{on} ~ \partial \Omega]
\end{align}

\begin{align}
\int_{\Omega}   w \xi^{n+1} ~dV&= \int_{\Omega}   w \xi^{n} - w \Delta t L_p~\left( m_p \left(  -2 \left( 1- \xi \right) \phi +2 \xi \left( 1- \phi \right)  \right) + m_g \left( -2 \left( 1-\xi \right) \sum_{i=1}^{n} \eta_i^2 \right) - \kappa_p \nabla^2 \xi \right)~dV \\
&= \int_{\Omega}   w ( \underbrace{ \eta^{n} - \Delta t L_p~\left( m_p \left(  -2 \left( 1- \xi \right) \phi +2 \xi \left( 1- \phi \right)  \right) + m_g \left( -2 \left( 1-\xi \right) \sum_{i=1}^{n} \eta_i^2 \right) \right) }_{r_{\eta_i}} + \grad w \underbrace{ (-\Delta t L_p \kappa_p)~ \cdot (\grad \xi^{n})}_{r_{\xi x}} ~dV \quad [\kappa_p \grad \xi \cdot n = 0 ~ \text{on} ~ \partial \Omega]
\end{align}

\begin{align}
\int_{\Omega}   w T^{n+1} ~dV &= \int_{\Omega}   w T^{n} + w (\Delta t D) \Delta T^n  \\
&= \int_{\Omega}   w (T^{n}) - \grad w  \cdot (\Delta t D) \grad T^n ~dV + \int_{\partial \Omega}   w  (\Delta t D) \grad T^n \cdot n ~dS\\
&= \int_{\Omega}   w (T^{n}) - \grad w  \cdot (\Delta t D) \grad T^n ~dV + \int_{\partial \Omega}   w  (\Delta t D) j^n  ~dS\\
&= \int_{\Omega}   w (\underbrace{T^{n}}_{r_c}) + \grad w  \cdot \underbrace{(-\Delta t D) \grad T^n}_{r_{cx}} ~dV \quad [\text {assuming flux}~j=0 ]
\end{align} 

\vskip 0.25in
The above values of  $r_{\eta_i}$ and $r_{\eta_i x}$ are used to define the residuals in the following parameters file: \\
\textit{applications/grainGrowth/equations.h}


\end{document} 
